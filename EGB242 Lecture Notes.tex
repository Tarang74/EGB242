%!TEX TS-program = xelatex
%!TEX options = -aux-directory=Debug -shell-escape -file-line-error -interaction=nonstopmode -halt-on-error -synctex=1 "%DOC%"
\documentclass{article}
\input{LaTeX-Submodule/template.tex}

% Additional packages & macros

% Header and footer
\newcommand{\unitName}{Signal Analysis}
\newcommand{\unitTime}{Semester 2, 2022}
\newcommand{\unitCoordinator}{Prof Wageeh Boles}
\newcommand{\documentAuthors}{Tarang Janawalkar}

\fancyhead[L]{\unitName}
\fancyhead[R]{\leftmark}
\fancyfoot[C]{\thepage}

% Copyright
\usepackage[
    type={CC},
    modifier={by-nc-sa},
    version={4.0},
    imagewidth={5em},
    hyphenation={raggedright}
]{doclicense}

\date{}

\begin{document}
%
\begin{titlepage}
    \vspace*{\fill}
    \begin{center}
        \LARGE{\textbf{\unitName}} \\[0.1in]
        \normalsize{\unitTime} \\[0.2in]
        \normalsize\textit{\unitCoordinator} \\[0.2in]
        \documentAuthors
    \end{center}
    \vspace*{\fill}
    \doclicenseThis
    \thispagestyle{empty}
\end{titlepage}
\newpage
%
\tableofcontents
\newpage
%
\section{Properties of Mathematical Functions}
\subsection{Even and Odd Functions}
\begin{definition}[Even function]
    A function \(x\left( t \right)\) is even if
    \begin{equation*}
        x\left( -t \right) = x\left( t \right)
    \end{equation*}
    for all \(t\) in the functions domain. Even functions are symmetric about the vertical axis.
\end{definition}
\begin{definition}[Odd function]
    A function \(x\left( t \right)\) is odd if
    \begin{equation*}
        x\left( -t \right) = -x\left( t \right)
    \end{equation*}
    for all \(t\) in the functions domain. Odd functions are symmetric about the origin.
\end{definition}
\subsubsection{Integrating Even and Odd Functions}
When integrating an \textbf{even} function \(x\left( t \right)\) over the domain \(\interval{-T}{T}\):
\begin{equation*}
    \int_{-T}^T x\left( t \right) \odif{t} = 2 \int_0^T x\left( t \right) \odif{t}.
\end{equation*}
Similarly, when integrating an \textbf{odd} function \(x\left( t \right)\) over the domain \(\interval{-T}{T}\):
\begin{equation*}
    \int_{-T}^T x\left( t \right) \odif{t} = 0.
\end{equation*}
\subsubsection{Product of Even and Odd Functions}
\begin{enumerate}
    \item The product of an \textbf{even} function with an \textbf{even} function, is an \textbf{even} function.

          Let \(f\left( t \right)\) and \(g\left( t \right)\) be even functions, and let \(h\left( t \right) = f\left( t \right) g\left( t \right)\),
          \begin{equation*}
              h\left( -t \right) = f\left( -t \right) g\left( -t \right) = f\left( t \right) g\left( t \right) = h\left( t \right).
          \end{equation*}
    \item The product of an \textbf{even} function with an \textbf{odd} function, is an \textbf{odd} function.

          Let \(f\left( t \right)\) be an even function and \(g\left( t \right)\) be an odd function, and let \(h\left( t \right) = f\left( t \right) g\left( t \right)\),
          \begin{equation*}
              h\left( -t \right) = f\left( -t \right) g\left( -t \right) = \left( -f\left( t \right) \right) g\left( t \right) = -h\left( t \right).
          \end{equation*}
    \item The product of an \textbf{odd} function with an \textbf{odd} function, is an \textbf{even} function.

          Let \(f\left( t \right)\) and \(g\left( t \right)\) be odd functions, and let \(h\left( t \right) = f\left( t \right) g\left( t \right)\),
          \begin{equation*}
              h\left( -t \right) = f\left( -t \right) g\left( -t \right) = \left( -f\left( t \right) \right) \left( -g\left( t \right) \right) = f\left( t \right) g\left( t \right) = h\left( t \right).
          \end{equation*}
\end{enumerate}
\subsection{Orthogonality}
\begin{definition}[Inner product]
    An inner product generalises the dot product in general vector spaces.

    In particular, for the function space \(\mathscr{F}\left( \interval{a}{b} \right)\), where \(t \in \interval{a}{b}\),
    the inner product is defined as the following:
    \begin{equation*}
        \abracket*{f,\: g} = \int_a^b f\left( t \right) g\left( t \right) \odif{t}
    \end{equation*}
    for \(f,\: g \in \mathscr{F}\left( \interval{a}{b} \right)\).
\end{definition}
\begin{definition}[Orthogonality]
    Given an inner product space, two vectors are orthogonal iff
    \begin{equation*}
        \abracket*{f,\: g} = 0.
    \end{equation*}
\end{definition}
\subsection{Orthogonality of Trigonometric Functions}
Consider the inner product between the sine and cosine functions on the interval \(\interval{-T}{T}\).
\begin{equation*}
    \abracket*{\sin{\left( t \right)},\: \cos{\left( t \right)}} = \int_{-T}^T \sin{\left( t \right)} \cos{\left( t \right)} \odif{t} = 0
\end{equation*}
as the integrand is an odd function.
\subsection{Integrals of Trigonometric Functions}
For \(n \in \Z\):
\begin{align*}
    \int_{t_0}^{t_0 + T} \sin{\left( 2\pi n f_0 t \right)} \odif{t} & = - \frac{1}{2\pi n f_0} \left[ \cos{\left( 2\pi n f_0 t \right)} \right]_{t_0}^{t_0 + T}                                                        \\
                                                                    & = - \frac{1}{2\pi n f_0} \left[ \cos{\left( \frac{2\pi n}{T} \left( t_0 + T \right) \right)} - \cos{\left( \frac{2\pi n}{T} t_0 \right)} \right] \\
                                                                    & = - \frac{1}{2\pi n f_0} \left[ \cos{\left( \frac{2\pi n}{T} t_0 + 2\pi n \right)} - \cos{\left( \frac{2\pi n}{T} t_0 \right)} \right]           \\
                                                                    & = - \frac{1}{2\pi n f_0} \left[ \cos{\left( \frac{2\pi n}{T} t_0 \right)} - \cos{\left( \frac{2\pi n}{T} t_0 \right)} \right]                    \\
                                                                    & = - \frac{1}{2\pi n f_0} \left[ 0 \right]                                                                                                        \\
                                                                    & = 0.
\end{align*}
\begin{align*}
    \int_{t_0}^{t_0 + T} \cos{\left( 2\pi n f_0 t \right)} \odif{t} & = \frac{1}{2\pi n f_0} \left[ \sin{\left( 2\pi n f_0 t \right)} \right]_{t_0}^{t_0 + T}                                                        \\
                                                                    & = \frac{1}{2\pi n f_0} \left[ \sin{\left( \frac{2\pi n}{T} \left( t_0 + T \right) \right)} - \sin{\left( \frac{2\pi n}{T} t_0 \right)} \right] \\
                                                                    & = \frac{1}{2\pi n f_0} \left[ \sin{\left( \frac{2\pi n}{T} t_0 + 2\pi n \right)} - \sin{\left( \frac{2\pi n}{T} t_0 \right)} \right]           \\
                                                                    & = \frac{1}{2\pi n f_0} \left[ \sin{\left( \frac{2\pi n}{T} t_0 \right)} - \sin{\left( \frac{2\pi n}{T} t_0 \right)} \right]                    \\
                                                                    & = \frac{1}{2\pi n f_0} \left[ 0 \right]                                                                                                        \\
                                                                    & = 0.
\end{align*}
\subsubsection{Product of Trigonometric Functions}
Recall the Werner formulas:
\begin{gather*}
    2 \cos{\left( \alpha \right)} \cos{\left( \beta \right)} = \cos{\left( \alpha - \beta \right)} + \cos{\left( \alpha + \beta \right)} \\
    2 \sin{\left( \alpha \right)} \sin{\left( \beta \right)} = \cos{\left( \alpha - \beta \right)} - \cos{\left( \alpha + \beta \right)} \\
    2 \sin{\left( \alpha \right)} \cos{\left( \beta \right)} = \sin{\left( \alpha - \beta \right)} + \sin{\left( \alpha + \beta \right)}
\end{gather*}
For \(n,\: m \in \N\),

\underline{Product of two cosine functions:}
\begin{equation*}
    \int_{t_0}^{t_0 + T} \cos{\left( 2\pi n f_0 t \right)} \cos{\left( 2\pi m f_0 t \right)} \odif{t} = \frac{1}{2} \int_{t_0}^{t_0 + T} \cos{\left( 2 \pi \left( n - m \right) f_0 t \right)} + \cos{\left( 2 \pi \left( n + m \right) f_0 t \right)} \odif{t}
\end{equation*}
\(n = m \implies n - m = 0\) and \(\left( n + m \right) \in \Z\), so that the integral of the second
term is \(0\), and the integral of the first term results in \(\frac{T}{2}\).

\(n \neq m \implies \left( n - m \right),\: \left( n + m \right) \in \Z\) so that both terms evaluate to \(0\) when integrated separately.
\begin{equation*}
    \int_{t_0}^{t_0 + T} \cos{\left( 2\pi n f_0 t \right)} \cos{\left( 2\pi m f_0 t \right)} \odif{t} = \begin{cases}
        \frac{T}{2}, & n = m     \\
        0,           & n \neq m.
    \end{cases}
\end{equation*}
\underline{Product of two sine functions:}
\begin{equation*}
    \int_{t_0}^{t_0 + T} \sin{\left( 2\pi n f_0 t \right)} \sin{\left( 2\pi m f_0 t \right)} \odif{t} = \frac{1}{2} \int_{t_0}^{t_0 + T} \cos{\left( 2 \pi \left( n - m \right) f_0 t \right)} - \cos{\left( 2 \pi \left( n + m \right) f_0 t \right)} \odif{t}
\end{equation*}
By the same argument as before,
\begin{equation*}
    \int_{t_0}^{t_0 + T} \sin{\left( 2\pi n f_0 t \right)} \sin{\left( 2\pi m f_0 t \right)} \odif{t} = \begin{cases}
        \frac{T}{2}, & n = m     \\
        0,           & n \neq m.
    \end{cases}
\end{equation*}
\underline{Product of sine and cosine functions:}
\begin{equation*}
    \int_{t_0}^{t_0 + T} \sin{\left( 2\pi n f_0 t \right)} \cos{\left( 2\pi m f_0 t \right)} \odif{t} = \frac{1}{2} \int_{t_0}^{t_0 + T} \sin{\left( 2 \pi \left( n - m \right) f_0 t \right)} + \sin{\left( 2 \pi \left( n + m \right) f_0 t \right)} \odif{t}
\end{equation*}
\(n = m \implies n - m = 0\) and \(\left( n + m \right) \in \Z\), so that the integral reduces to \(0\).

\(n \neq m \implies \left( n - m \right),\: \left( n + m \right) \in \Z\) so that both terms evaluate to \(0\) when integrated separately.
\begin{equation*}
    \int_{t_0}^{t_0 + T} \sin{\left( 2\pi n f_0 t \right)} \cos{\left( 2\pi m f_0 t \right)} \odif{t} = 0.
\end{equation*}
\section{Fourier Series}
\subsection{Fourier Series Expansion}
The \textbf{Fourier Series Expansion} of a function \(x\left( t \right)\) on the interval \(\interval{t_0}{t_0 + T}\)
is given by
\begin{equation*}
    x_F\left( t \right) = a_0 + \sum_{n = 1}^\infty \left( a_n \cos{\left( 2\pi n f_0 t \right)} + b_n \sin{\left( 2\pi n f_0 t \right)} \right)
\end{equation*}
where \(n \in \Z^{+}\) and \(f_0 = \frac{1}{T}\). The coefficients are given by
\begin{align*}
    a_0 & = \frac{1}{T} \int_{t_0}^{t_0 + T} x\left( t \right) \odif{t}                                   \\
    a_n & = \frac{2}{T} \int_{t_0}^{t_0 + T} x\left( t \right) \cos{\left( 2\pi n f_0 t \right)} \odif{t} \\
    b_n & = \frac{2}{T} \int_{t_0}^{t_0 + T} x\left( t \right) \sin{\left( 2\pi n f_0 t \right)} \odif{t}
\end{align*}
\begin{proof}
    Let \(m \in \N\).

    For the coefficient \(a_0\), integrate the function \(x\left( t \right)\) over the interval \(\interval{t_0}{t_0 + T}\).
    \begin{align*}
        \int_{t_0}^{t_0 + T} x\left( t \right) \odif{t} & = \int_{t_0}^{t_0 + T} a_0 \odif{t} + \sum_{n = 1}^\infty a_n \cancelto{0}{\int_{t_0}^{t_0 + T} \cos{\left( 2 \pi n f_0 t \right)} \odif{t}} + \sum_{n = 1}^\infty b_n \cancelto{0}{\int_{t_0}^{t_0 + T} \sin{\left( 2 \pi n f_0 t \right)} \odif{t}} \\
        \int_{t_0}^{t_0 + T} x\left( t \right) \odif{t} & = a_0 T                                                                                                                                                                                                                                               \\
        a_0                                             & = \frac{1}{T} \int_{t_0}^{t_0 + T} x\left( t \right) \odif{t}
    \end{align*}
    so that \(a_0\) represents the average value of \(x\) on \(\interval{t_0}{t_0 + T}\). This coefficient also
    represents the DC component of a signal.

    For coefficients \(a_m\), multiply the equation by \(\cos{\left( 2 \pi m f_0 t \right)}\) before integrating.
    \begin{align*}
        x\left( t \right) \cos{\left( 2 \pi m f_0 t \right)}                               & = \begin{aligned}[t]
                                                                                                    & a_0 \cos{\left( 2 \pi m f_0 t \right)}                                                          \\
                                                                                                    & + \sum_{n = 1}^\infty a_n \cos{\left( 2 \pi n f_0 t \right)} \cos{\left( 2 \pi m f_0 t \right)} \\
                                                                                                    & + \sum_{n = 1}^\infty b_n \sin{\left( 2 \pi n f_0 t \right)} \cos{\left( 2 \pi m f_0 t \right)}
                                                                                               \end{aligned}                                                    \\
        \int_{t_0}^{t_0 + T} x\left( t \right) \cos{\left( 2 \pi m f_0 t \right)} \odif{t} & = \begin{aligned}[t]
                                                                                                    & a_0 \cancelto{0}{\int_{t_0}^{t_0 + T} \cos{\left( 2 \pi m f_0 t \right)} \odif{t}}                                                          \\
                                                                                                    & + \sum_{n = 1}^\infty a_n \int_{t_0}^{t_0 + T} \cos{\left( 2 \pi n f_0 t \right)} \cos{\left( 2 \pi m f_0 t \right)} \odif{t}               \\
                                                                                                    & + \sum_{n = 1}^\infty b_n \cancelto{0}{\int_{t_0}^{t_0 + T} \sin{\left( 2 \pi n f_0 t \right)} \cos{\left( 2 \pi m f_0 t \right)} \odif{t}}
                                                                                               \end{aligned} \\
        \int_{t_0}^{t_0 + T} x\left( t \right) \cos{\left( 2 \pi m f_0 t \right)} \odif{t} & = a_m \frac{T}{2}                                                                                                                                       \\
        a_m                                                                                & = \frac{2}{T} \int_{t_0}^{t_0 + T} x\left( t \right) \cos{\left( 2 \pi m f_0 t \right)} \odif{t}
    \end{align*}

    For coefficients \(b_m\), multiply the equation by \(\sin{\left( 2 \pi m f_0 t \right)}\) before integrating.
    \begin{align*}
        x\left( t \right) \sin{\left( 2 \pi m f_0 t \right)}                               & = \begin{aligned}[t]
                                                                                                    & a_0 \sin{\left( 2 \pi m f_0 t \right)}                                                          \\
                                                                                                    & + \sum_{n = 1}^\infty a_n \cos{\left( 2 \pi n f_0 t \right)} \sin{\left( 2 \pi m f_0 t \right)} \\
                                                                                                    & + \sum_{n = 1}^\infty b_n \sin{\left( 2 \pi n f_0 t \right)} \sin{\left( 2 \pi m f_0 t \right)}
                                                                                               \end{aligned}                                                    \\
        \int_{t_0}^{t_0 + T} x\left( t \right) \sin{\left( 2 \pi m f_0 t \right)} \odif{t} & = \begin{aligned}[t]
                                                                                                    & a_0 \cancelto{0}{\int_{t_0}^{t_0 + T} \sin{\left( 2 \pi m f_0 t \right)} \odif{t}}                                                          \\
                                                                                                    & + \sum_{n = 1}^\infty a_n \cancelto{0}{\int_{t_0}^{t_0 + T} \cos{\left( 2 \pi n f_0 t \right)} \sin{\left( 2 \pi m f_0 t \right)} \odif{t}} \\
                                                                                                    & + \sum_{n = 1}^\infty b_n \int_{t_0}^{t_0 + T} \sin{\left( 2 \pi n f_0 t \right)} \sin{\left( 2 \pi m f_0 t \right)} \odif{t}
                                                                                               \end{aligned} \\
        \int_{t_0}^{t_0 + T} x\left( t \right) \sin{\left( 2 \pi m f_0 t \right)} \odif{t} & = b_m \frac{T}{2}                                                                                                                                       \\
        b_m                                                                                & = \frac{2}{T} \int_{t_0}^{t_0 + T} x\left( t \right) \sin{\left( 2 \pi m f_0 t \right)} \odif{t}
    \end{align*}
\end{proof}
\subsubsection{Convergence of a Fourier Series}
If \(x\left( t \right)\) is piecewise smooth on \(\interval{t_0}{t_0 + L}\), \(x_F\left( t \right)\)
converges to
\begin{equation*}
    x_F\left( t \right) = \lim_{\epsilon \to 0^{+}} \frac{x\left( t + \epsilon \right) + x\left( t - \epsilon \right)}{2}
\end{equation*}
that is, \(x = x_F\), except at discontinuities, where \(f_F\) is equal to the point halfway between the left- and right-handed limits.
\subsubsection{Periodicity of a Fourier Series}
If \(x\) is non-periodic, \(x_F\) converges to the periodic extension of \(x\). The endpoints may converge non-uniformly,
corresponding to jump discontinuities in the periodic extension of \(x\).
\subsection{Fourier Cosine Series}
Consider the Fourier series expansion of an even function \(x\) on the interval \(\interval{-\frac{T}{2}}{\frac{T}{2}}\), i.e., \(t_0 = -\frac{T}{2}\).
In this case,
\begin{equation*}
    b_n = \frac{2}{T} \int_{-\frac{T}{2}}^{\frac{T}{2}} x\left( t \right) \sin{\left( 2\pi n f_0 t \right)} \odif{t} = 0
\end{equation*}
and the Fourier series is a ``Fourier cosine series'', given by:
\begin{equation*}
    x_c\left( t \right) = a_0 + \sum_{n = 1}^\infty a_n \cos{\left( 2\pi n f_0 t \right)}
\end{equation*}
with coefficients
\begin{align*}
    a_0 & = \frac{1}{T} \int_{-\frac{T}{2}}^{\frac{T}{2}} x\left( t \right) \odif{t}                                    = \frac{2}{T} \int_0^{\frac{T}{2}} x\left( t \right) \odif{t}                                    \\
    a_n & = \frac{2}{T} \int_{-\frac{T}{2}}^{\frac{T}{2}} x\left( t \right) \cos{\left( 2 \pi n f_0 t \right)} \odif{t} = \frac{4}{T} \int_0^{\frac{T}{2}} x\left( t \right) \cos{\left( 2\pi n f_0 t \right)} \odif{t}.
\end{align*}
\subsection{Fourier Sine Series}
Consider the Fourier series expansion of an odd function \(x\) on the interval \(\interval{-\frac{T}{2}}{\frac{T}{2}}\).
In this case
\begin{equation*}
    b_n = \frac{2}{T} \int_{-\frac{T}{2}}^{\frac{T}{2}} x\left( t \right) \sin{\left( 2\pi n f_0 t \right)} \odif{t} = 0
\end{equation*}
and the Fourier series is a ``Fourier sine series'', given by:
\begin{equation*}
    x_s\left( t \right) = \sum_{n = 1}^\infty b_n \sin{\left( 2\pi n f_0 t \right)}
\end{equation*}
with coefficients
\begin{equation*}
    b_n = \frac{2}{T} \int_{-\frac{T}{2}}^{\frac{T}{2}} x\left( t \right) \sin{\left( 2 \pi n f_0 t \right)} \odif{t} = \frac{4}{T} \int_0^{\frac{T}{2}} x\left( t \right) \sin{\left( 2\pi n f_0 t \right)} \odif{t}.
\end{equation*}
\section{Complex Fourier Series}
\begin{definition}
    The \textbf{Complex Fourier Series Expansion} is a concise form of the Fourier series expansion that uses complex exponentials with a single unknown coefficient.
    \begin{equation*}
        x_C\left( t \right) = \sum_{n = -\infty}^\infty c_n e^{j 2\pi n f_0 t}
    \end{equation*}
    where
    \begin{equation*}
        c_n = \frac{1}{T} \int_{t_0}^{t_0 + T} x\left( t \right) e^{-j 2\pi n f_0 t} \odif{t}.
    \end{equation*}
    for \(n \in \Z\) and \(f_0 = \frac{1}{T}\).
\end{definition}
To determine the complex Fourier series expansion consider the following identities:
\begin{align*}
    \cos{\left( \theta \right)} & = \frac{e^{j \theta} + e^{-j \theta}}{2}     \\
    \sin{\left( \theta \right)} & = -j \frac{e^{j \theta} - e^{-j \theta}}{2}.
\end{align*}
By substituting these identities into the Fourier series expansion summand, we obtain:
\begin{align*}
    a_n \cos{\left( 2\pi n f_0 t \right)} + b_n \sin{\left( 2\pi n f_0 t \right)} & = a_n \frac{e^{j 2\pi n f_0 t} + e^{-j 2\pi n f_0 t}}{2} -j b_n \frac{e^{j 2\pi n f_0 t} - e^{-j 2\pi n f_0 t}}{2} \\
                                                                                  & = \frac{a_n - j b_n}{2} e^{j 2\pi n f_0 t} + \frac{a_n + j b_n}{2} e^{-j 2\pi n f_0 t}
\end{align*}
Let \(c_n = \frac{a_n - j b_n}{2}\) and \(c_n^\ast = \frac{a_n + j b_n}{2}\) (we will see how this simplifies later).
Using the definitions for \(a_n\) and \(b_n\):
\begin{align*}
    c_n & = \frac{1}{2} \left( a_n - j b_n \right)                                                                                                            \\
        & = \frac{1}{T} \int_{t_0}^{t_0 + T} x\left( t \right) \left( \cos{\left( 2\pi n f_0 t \right)} -j \sin{\left( 2\pi n f_0 t \right)} \right) \odif{t} \\
        & = \frac{1}{T} \int_{t_0}^{t_0 + T} x\left( t \right) e^{-j 2\pi n f_0 t} \odif{t}
\end{align*}
\begin{align*}
    c_n^\ast & = \frac{1}{2} \left( a_n + j b_n \right)                                                                                                             \\
             & = \frac{1}{T} \int_{t_0}^{t_0 + T} x\left( t \right) \left( \cos{\left( 2\pi n f_0 t \right)} + j \sin{\left( 2\pi n f_0 t \right)} \right) \odif{t} \\
             & = \frac{1}{T} \int_{t_0}^{t_0 + T} x\left( t \right) e^{j 2\pi n f_0 t} \odif{t}                                                                     \\
             & = c_{-n}
\end{align*}
Let \(c_0 = a_0\), so that
\begin{align*}
    x_C\left( t \right) & = a_0 + \sum_{n = 1}^\infty \left( a_n \cos{\left( 2\pi n f_0 t \right)} + b_n \sin{\left( 2\pi n f_0 t \right)} \right) \\
                        & = c_0 + \sum_{n = 1}^\infty \left( c_n e^{j 2\pi n f_0 t} + c_{-n} e^{-j 2\pi n f_0 t} \right)                           \\
                        & = c_0 + \sum_{n = 1}^\infty c_n e^{j 2\pi n f_0 t} + \sum_{n = 1}^\infty c_{-n} e^{-j 2\pi n f_0 t}                      \\
                        & = c_0 + \sum_{n = 1}^\infty c_n e^{j 2\pi n f_0 t} + \sum_{n = -\infty}^{-1} c_{n} e^{j 2\pi n f_0 t}                    \\
                        & = \sum_{n = -\infty}^\infty c_n e^{j 2\pi n f_0 t}.
\end{align*}
\subsection{Converting between Fourier Series Representations}
Given the Trigonometric and Exponential (Complex) Fourier Series Representations (FSR),
we can develop a relationship between the coefficients \(a_n\), \(b_n\), and \(c_n\) by:
\begin{align*}
    a_0 & = c_0                            \\
    a_n & = c_n + c_{-n}                   \\
    b_n & = j \left( c_n - c_{-n} \right).
\end{align*}
\subsection{Magnitude and Phase Spectra}
As \(c_n\) is a complex number, consider the polar representation of \(c_n\):
\begin{equation*}
    c_n = \abs*{c_n} e^{j \theta_n}
\end{equation*}
so that the \textbf{magnitude spectra} is given by \(\abs*{c_n}\) and the \textbf{phase spectra} is given by \(\theta_n\).

The plot of \(\abs*{c_n}\) against \(n\) is called the ``magnitude spectrum'' of \(x\left( t \right)\) and
the plot of \(\theta_n\) against \(n\) is called the ``phase spectrum'' of \(x\left( t \right)\).
\begin{theorem}[Spectra of a real signal]
    Given any real function \(x\left( t \right)\), the magnitude spectrum is always an even function,
    and the phase spectrum is always an odd function.
\end{theorem}
\begin{proof}
    Given a real function \(x\left( t \right)\), the exponential Fourier series is given by,
    \begin{equation*}
        x_C\left( t \right) = \sum_{n = -\infty}^\infty c_n e^{j 2\pi n f_0 t}
    \end{equation*}
    this is equivalent to
    \begin{equation*}
        x_C\left( t \right) = \sum_{n = -\infty}^\infty c_{-n} e^{-j 2\pi n f_0 t}.
    \end{equation*}
    The conjugate of \(x_C\left( t \right)\) yields,
    \begin{equation*}
        \overline{x_C\left( t \right)} = \sum_{n = -\infty}^\infty \overline{c_{n}} e^{-j 2\pi m f_0 t}
    \end{equation*}
    As \(x\left( t \right) \in \R\), \(x_C\left( t \right) = \overline{x_C\left( t \right)}\), so that
    \begin{align*}
        \sum_{n = -\infty}^\infty c_{-n} e^{-j 2\pi n f_0 t} & = \sum_{n = -\infty}^\infty \overline{c_{n}} e^{-j 2\pi m f_0 t} \\
        c_{-n} e^{-j 2\pi n f_0 t}                           & = \overline{c_n} e^{-j 2\pi n f_0 t}                             \\
        c_{-n}                                               & = \overline{c_n}
    \end{align*}
    Therefore by representing the coefficients above in polar form we get,
    \begin{equation*}
        \abs*{c_{-n}} e^{j \theta_{-n}} = \abs*{c_n} e^{-j \theta_n}
    \end{equation*}
    as required.
\end{proof}
\subsection{Even and Odd Functions}
Given an even function \(x\left( t \right)\), \(b_n = 0\), therefore
\begin{align*}
    0   & = j \left( c_n - c_{-n} \right) \\
    c_n & = c_{-n}
\end{align*}
so that,
\begin{align*}
    a_n & = c_n + c_{-n}           \\
    a_n & = 2 c_n                  \\
    c_n & = c_{-n} = \frac{a_n}{2}
\end{align*}
Hence \(c_n\) and \(c_{-n}\) are real coefficients, so that \(\abs{c_n}\) is an even function, and \(\theta_n = m \pi\) for some \(m \in \Z\).

Given an odd function \(x\left( t \right)\), \(a_0 = 0\) and \(a_n = 0\), therefore \(c_0 = 0\) and
\begin{align*}
    0   & = c_n + c_{-n} \\
    c_n & = -c_{-n}
\end{align*}
so that,
\begin{align*}
    b_n    & = j \left( c_n - c_{-n} \right) \\
    b_n    & = j \left( c_n + c_n \right)    \\
    b_n    & = j 2 c_n                       \\
    c_n    & = -j \frac{b_n}{2}              \\
    c_{-n} & = j \frac{b_n}{2}
\end{align*}
Hence \(c_n\) and \(c_{-n}\) are purely imaginary coefficients, so that \(\abs{c_n}\) is an odd function, and \(\theta_n = \left( 2m + 1 \right) \frac{\pi}{2}\) for some \(m \in \Z\).
\subsection{Signal Representations}
A signal can be represented in various forms depending on the method of measurement. There are three main forms of signal representation:
\begin{itemize}
    \item Analogue: Continuous in time and continuous in amplitude.
    \item Discrete: Discrete in time and continuous in amplitude.
    \item Digital: Discrete in time and discrete in amplitude.
\end{itemize}
The process of taking discrete time measurements is known as sampling, and taking discrete amplitude measurements is known as quantisation.
\begin{definition}[Integral transform]
    An \textit{integral transform} transforms a function through the process of integration, producing a new function of another variable.
    Two common examples of integral transforms are the Fourier transform and the Laplace transform.
\end{definition}
\section{Fourier Transform}
The Fourier Transform allows us to extend the techniques used in Fourier series representations of functions
to non-periodic signals.
\begin{definition}[Fourier transform]
    The Fourier transform of a function \(x\left( t \right)\) is defined by:
    \begin{equation*}
        \mathscr{F}\left\{ x\left( t \right) \right\} = X\left( f \right) = \int_{-\infty}^\infty x\left( t \right) e^{-j 2\pi f t} \odif{t}.
    \end{equation*}
    where \(X\left( f \right) = \mathscr{F}\left\{ x\left( t \right) \right\}\) is a function of frequency \(f\).
\end{definition}
\begin{definition}[Inverse Fourier transform]
    The inverse Fourier transform of a function \(X\left( f \right)\) is defined by:
    \begin{equation*}
        \mathscr{F}^{-1}\left\{ X\left( f \right) \right\} = x\left( t \right) = \int_{-\infty}^\infty X\left( f \right) e^{j 2\pi f t} \odif{f}.
    \end{equation*}
\end{definition}
This integral transform is commonly represented as a ``Fourier pair'' using the following notation:
\begin{equation*}
    x\left( t \right) \overset{\mathscr{F}}{\longleftrightarrow} X\left( f \right).
\end{equation*}
We can derive this definition by considering the complex Fourier series representation of \(x\left( t \right)\)
on the interval \(\interval{-\frac{T}{2}}{\frac{T}{2}}\):
\begin{align*}
    x\left( t \right) & = \sum_{n = -\infty}^\infty c_n e^{j 2\pi n f_0 t}                                                                                                          \\
                      & = \sum_{n = -\infty}^\infty \left[ \frac{1}{T} \int_{-\frac{T}{2}}^{\frac{T}{2}} x\left( t \right) e^{-j 2\pi n f_0 t} \odif{t} \right] e^{j 2\pi n f_0 t}.
\end{align*}
If we let \(f_n = n f_0\), then \(\adif{f_n} = f_{n + 1} - f_n = \left( n + 1 \right)f_0 - n f_0 = f_0 = \frac{1}{T}\):
\begin{equation*}
    x\left( t \right) = \sum_{n = -\infty}^\infty \left[ \int_{-\frac{T}{2}}^{\frac{T}{2}} x\left( t \right) e^{-j 2\pi f_n t} \odif{t} \right] e^{j 2\pi f_n t} \adif{f_n}.
\end{equation*}
By recognition, this is a Riemann sum, so that by taking the limit \(T \to \infty\) we get:
\begin{equation*}
    x\left( t \right) = \int_{-\infty}^\infty \left[ \int_{-\infty}^\infty x\left( t \right) e^{-j 2\pi f t} \odif{t} \right] e^{j 2\pi f t} \odif{f}
\end{equation*}
where
\begin{equation*}
    X\left( f \right) = \int_{-\infty}^\infty x\left( t \right) e^{-j 2\pi f t} \odif{t}.
\end{equation*}
\begin{corollary}[Dirichlet conditions]
    The Dirichlet conditions provide sufficient conditions for a real-valued function \(x\) to be
    equal to its Fourier Transform \(X\), at each point where \(x\) is continuous.

    The conditions are:
    \begin{enumerate}
        \item \(x\) has a finite number of maxima and minima over \(\interval{-L}{L}\).
        \item \(x\) has a finite number of discontinuities, in each of which the derivative \(x'\) exists and does not change sign.
        \item \(\int_{-\infty}^\infty \abs*{x\left( t \right)} \odif{t}\) exists.
    \end{enumerate}
\end{corollary}
\subsection{Fourier Transform Properties}
\subsubsection{Linearity}
\begin{equation*}
    a_1 x_1\left( t \right) \pm a_2 x_2\left( t \right) \overset{\mathscr{F}}{\longleftrightarrow} a_1 X_1\left( f \right) \pm a_2 X_2\left( f \right).
\end{equation*}
Due to the linearity of the integral, the Fourier transform is a linear operator.
\subsubsection{Complex Conjugate}
\begin{equation*}
    x^\ast\left( t \right) \overset{\mathscr{F}}{\longleftrightarrow} X^\ast\left( -f \right).
\end{equation*}
\begin{proof}
    \begin{align*}
        \mathscr{F}\left\{ x^\ast\left( t \right) \right\} & = \int_{-\infty}^\infty x^\ast\left( t \right) e^{-j 2\pi f t} \odif{t}                       \\
                                                           & = \int_{-\infty}^\infty x^\ast\left( t \right) \overline{e^{j 2\pi f t}} \odif{t}^\ast        \\
                                                           & = \int_{-\infty}^\infty \overline{x\left( t \right) e^{j 2\pi f t} \odif{t}}                  \\
                                                           & = \overline{\int_{-\infty}^\infty x\left( t \right) e^{j 2\pi f t} \odif{t}}                  \\
                                                           & = \overline{\int_{-\infty}^\infty x\left( t \right) e^{-j 2\pi \left( -f \right) t} \odif{t}} \\
                                                           & = X^\ast\left( -f \right).
    \end{align*}
\end{proof}
\subsubsection{Time Shift}
\begin{equation*}
    x\left( t - t_0 \right) \overset{\mathscr{F}}{\longleftrightarrow} e^{-j 2\pi f t_0} X\left( f \right).
\end{equation*}
\begin{proof}
    \begin{align*}
        x\left( t - t_0 \right) & = \int_{-\infty}^\infty X\left( f \right) e^{j 2\pi f \left( t - t_0 \right)} \odif{f}              \\
                                & = \int_{-\infty}^\infty \left[ e^{-j 2\pi f t_0} X\left( f \right) \right] e^{-j 2\pi f t} \odif{f} \\
                                & = \mathscr{F}^{-1}{\left( e^{-j 2\pi f t_0} X\left( f \right) \right)}.
    \end{align*}
\end{proof}
\subsection{Frequency Shift}
\begin{equation*}
    e^{j 2\pi f_0 t} x\left( t \right) \overset{\mathscr{F}}{\longleftrightarrow} X\left( f - f_0 \right).
\end{equation*}
\begin{proof}
    \begin{align*}
        X\left( f - f_0 \right) & = \int_{-\infty}^\infty x\left( t \right) e^{-j 2\pi \left( f - f_0 \right) t} \odif{t}            \\
                                & = \int_{-\infty}^\infty \left[ e^{j 2\pi f_0 t} x\left( t \right) \right] e^{-j 2\pi f t} \odif{t} \\
                                & = \mathscr{F}\left\{ e^{j 2\pi f_0 t} x\left( t \right) \right\}
    \end{align*}
\end{proof}
\subsection{Time Reversal}
\begin{equation*}
    x\left( -t \right) \overset{\mathscr{F}}{\longleftrightarrow} X\left( -f \right).
\end{equation*}
\begin{proof}
    Consider the substitution \(u = -t\) so that \(\odif{u} = -\odif{t}\):
    \begin{align*}
        \mathscr{F}\left\{ x\left( -t \right) \right\} & = \int_{-\infty}^\infty x\left( -t \right) e^{-j 2\pi f t} \odif{t}                \\
                                                       & = -\int_\infty^{-\infty} x\left( u \right) e^{j 2\pi f u} \odif{u}                 \\
                                                       & = \int_{-\infty}^\infty x\left( u \right) e^{-j 2\pi \left( -f \right) u} \odif{u} \\
                                                       & = X\left( -f \right)
    \end{align*}
\end{proof}
\subsection{Time Scaling}
\begin{equation*}
    x\left( at \right) \overset{\mathscr{F}}{\longleftrightarrow} \frac{1}{\abs*{a}} X\left( \frac{f}{a} \right).
\end{equation*}
\begin{proof}
    Consider the substitution \(u = at\) so that \(\odif{u} = a \odif{t}\):
    \begin{align*}
        \mathscr{F}\left\{ x\left( at \right) \right\} & = \int_{-\infty}^\infty x\left( at \right) e^{-j 2\pi f t} \odif{t}                      \\
                                                       & = \int_{-\infty}^\infty x\left( u \right) e^{-j 2\pi f \frac{u}{a}} \frac{1}{a} \odif{u} \\
                                                       & = \frac{1}{a} \int_{-\infty}^\infty x\left( u \right) e^{-j 2\pi \frac{f}{a} u} \odif{u} \\
                                                       & = \frac{1}{a} X\left( \frac{f}{a} \right)
    \end{align*}
    when \(a > 0\), \(a = \abs*{a}\):
    \begin{equation*}
        \mathscr{F}\left\{ x\left( at \right) \right\} = \frac{1}{\abs*{a}} X\left( \frac{f}{a} \right).
    \end{equation*}
    When \(a < 0\), \(a = -\abs*{a}\):
    \begin{align*}
        \mathscr{F}\left\{ x\left( -\abs*{a} t \right) \right\} & = \frac{1}{\abs*{a}} X\left( -\frac{f}{\abs*{a}} \right) \\
                                                                & = \frac{1}{\abs*{a}} X\left( \frac{f}{a} \right).
    \end{align*}
\end{proof}
\subsection{Time Differentiation}
\begin{equation*}
    \odv[order=n]{x\left( t \right)}{t} \overset{\mathscr{F}}{\longleftrightarrow} \left( j 2\pi f \right)^n X\left( f \right).
\end{equation*}
\begin{proof}
    Consider the representation of \(\odv{x\left( t \right)}{t}\) using the inverse Fourier transform:
    \begin{align*}
        \odv{x\left( t \right)}{t} & = \odv*{\int_{-\infty}^\infty X\left( f \right) e^{j 2\pi f t}\odif{f}}{t} \\
                                   & = \int_{-\infty}^\infty X\left( f \right) \odv*{e^{j 2\pi f t}}{t}\odif{f} \\
                                   & = j 2\pi f \int_{-\infty}^\infty X\left( f \right) e^{j 2\pi f t}\odif{f}  \\
                                   & = j 2\pi f x\left( t \right)
    \end{align*}
    then the Fourier transform of \(\odv{x\left( t \right)}{t}\) is given by:
    \begin{align*}
        \mathscr{F}\left\{ \odv{x\left( t \right)}{t} \right\} & = \mathscr{F}\left\{ j 2\pi f x\left( t \right) \right\} \\
                                                               & = j 2\pi f X\left( f \right)
    \end{align*}
    Repeated differentiation yields the following result:
    \begin{equation*}
        \mathscr{F}\left\{ \odv[order=n]{x\left( t \right)}{t} \right\} = \left( j 2\pi f \right)^n X\left( f \right)
    \end{equation*}
\end{proof}
\subsection{Frequency Differentiation}
\begin{equation*}
    \left( \frac{2\pi}{j} \right)^n t^n x\left( t \right) \overset{\mathscr{F}}{\longleftrightarrow} \odv[order=n]{X\left( f \right)}{f}.
\end{equation*}
\begin{proof}
    Consider the representation of \(\odv{X\left( f \right)}{f}\) using the Fourier transform:
    \begin{align*}
        \odv{X\left( f \right)}{f} & = \odv*{\int_{-\infty}^\infty x\left( t \right) e^{-j 2\pi f t}\odif{t}}{f} \\
                                   & = \int_{-\infty}^\infty x\left( t \right) \odv*{e^{-j 2\pi f t}}{f}\odif{t} \\
                                   & = -j 2\pi t \int_{-\infty}^\infty x\left( t \right) e^{-j 2\pi f t}\odif{t} \\
                                   & = -j 2\pi t X\left( f \right)
    \end{align*}
    then the inverse Fourier transform of \(\odv{X\left( f \right)}{f}\) is given by:
    \begin{align*}
        \mathscr{F}^{-1}\left\{ \odv{X\left( f \right)}{f} \right\} & = \mathscr{F}^{-1}\left\{ -j 2\pi t X\left( f \right) \right\} \\
                                                                    & = -j 2\pi t \mathscr{F}^{-1}\left\{ X\left( f \right) \right\} \\
                                                                    & = \frac{2\pi}{j} t x\left( t \right)
    \end{align*}
    hence
    \begin{equation*}
        \frac{2\pi}{j} t x\left( t \right) \overset{\mathscr{F}}{\longleftrightarrow} \odv{X\left( f \right)}{f}.
    \end{equation*}
    Repeated differentiation yields the required result.
\end{proof}
\subsubsection{Time Multiplication}
\begin{equation*}
    t^n x\left( t \right) \overset{\mathscr{F}}{\longleftrightarrow} \left( \frac{j}{2\pi} \right)^n \odv[order=n]{X\left( f \right)}{f}.
\end{equation*}
\textit{See the previous section for the proof.}
\subsubsection{Time Integration}
\begin{equation*}
    \int_{-\infty}^t x\left( \tau \right) \odif{\tau} \overset{\mathscr{F}}{\longleftrightarrow} \frac{1}{j 2\pi f} X\left( f \right).
\end{equation*}
\begin{proof}
    Consider the representation of \(\int_{-\infty}^t x\left( \tau \right) \odif{\tau}\) using the inverse Fourier transform:
    \begin{align*}
        x\left( t \right)                                 & = \int_{-\infty}^\infty X\left( f \right) e^{j 2\pi f t} \odif{f}                                                  \\
        \int_{-\infty}^t x\left( \tau \right) \odif{\tau} & = \int_{-\infty}^t \left[ \int_{-\infty}^\infty X\left( f \right) e^{j 2\pi f \tau} \odif{f} \right] \odif{\tau}   \\
                                                          & = \int_{-\infty}^\infty X\left( f \right) \left[ \int_{-\infty}^t e^{j 2\pi f \tau} \odif{\tau} \right] \odif{f}   \\
                                                          & = \int_{-\infty}^\infty X\left( f \right) \left[ \frac{1}{j 2\pi f} e^{j 2\pi f \tau} \right]_{-\infty}^t \odif{f} \\
                                                          & = \frac{1}{j 2\pi f} \int_{-\infty}^\infty X\left( f \right) \left[ e^{j 2\pi f \tau} \right]_{-\infty}^t \odif{f} \\
                                                          & = \frac{1}{j 2\pi f} \int_{-\infty}^\infty X\left( f \right) e^{j 2\pi f t} \odif{f}                               \\
                                                          & = \frac{1}{j 2\pi f} x\left( t \right).
    \end{align*}
    We can now take the Fourier transform of \(\int_{-\infty}^t x\left( \tau \right) \odif{\tau}\) to obtain:
    \begin{align*}
        \mathscr{F}\left\{ \int_{-\infty}^t x\left( \tau \right) \odif{\tau} \right\} & = \mathscr{F}\left\{ \frac{1}{j 2\pi f} x\left( t \right) \right\} \\
                                                                                      & = \frac{1}{j 2\pi f} \mathscr{F}\left\{ x\left( t \right) \right\} \\
                                                                                      & = \frac{1}{j 2\pi f} X\left( f \right).
    \end{align*}
\end{proof}
To summarise:
\begin{align*}
    x\left( t \right)                                 & \overset{\mathscr{F}}{\leftrightarrow} X\left( f \right)                                                                                            \\
    x^\ast\left( t \right)                            & \overset{\mathscr{F}}{\longleftrightarrow} X^\ast\left( -f \right)                                                  &  & \text{(complex-conjugate)} \\
    x\left( t - t_0 \right)                           & \overset{\mathscr{F}}{\leftrightarrow} e^{-j 2\pi n f t_0} X\left( f \right)                                        &  & \text{(time-shift)}        \\
    e^{j 2\pi n f_0 t} x\left( t \right)              & \overset{\mathscr{F}}{\leftrightarrow} X\left( f - f_0 \right)                                                      &  & \text{(frequency-shift)}   \\
    x\left( -t \right)                                & \overset{\mathscr{F}}{\leftrightarrow} X\left( -f \right)                                                           &  & \text{(time reversing)}    \\
    x\left( at \right)                                & \overset{\mathscr{F}}{\leftrightarrow} \frac{1}{\abs*{a}} X\left( \frac{f}{a} \right)                               &  & \text{(time scaling)}      \\
    t x\left( t \right)                               & \overset{\mathscr{F}}{\leftrightarrow} \frac{j}{2\pi} \odv{X\left( f \right)}{f}                                                                    \\
    e^{at} u\left( t \right)                          & \overset{\mathscr{F}}{\leftrightarrow} \frac{1}{a + j 2\pi f}                                                                                       \\
    x\left( t \right) \cos{\left( 2\pi f_0 t \right)} & \overset{\mathscr{F}}{\leftrightarrow} \frac{1}{2} \left[ X\left( f - f_0 \right) + X\left( f + f_0 \right) \right] &  & \text{(time modulation)}
\end{align*}
\section{Special Functions}
\subsection{Dirac Delta Function}
The Dirac delta (or impulse) function is defined by the following characteristics:
\begin{align*}
    \delta\left( t \right)                                & = \begin{cases}
                                                                  0,      & t \neq 0 \\
                                                                  \infty, & t = 0
                                                              \end{cases} \\
    \int_{-\infty}^\infty \delta\left( t \right) \odif{t} & = 1.
\end{align*}
Likewise, the shifted impulse function can be defined as:
\begin{align*}
    \delta\left( t - t_0 \right)                                & = \begin{cases}
                                                                        0,      & t_0 \neq 0 \\
                                                                        \infty, & t_0 = 0
                                                                    \end{cases} \\
    \int_{-\infty}^\infty \delta\left( t - t_0 \right) \odif{t} & = 1.
\end{align*}
Therefore we can infer the following \textit{shifting properties}:
\begin{equation*}
    \int_{-\infty}^\infty f\left( t \right) \delta\left( t - t_0 \right) \odif{t} = f\left( t_0 \right)
\end{equation*}
and
\begin{equation*}
    f\left( t \right) \delta\left( t - t_0 \right) = f\left( t_0 \right) \delta\left( t - t_0 \right).
\end{equation*}
\subsection{Signum Function}
The signum function is defined as:
\begin{equation*}
    \mathrm{sgn}\left( t \right) = \begin{cases}
        1  & x > 0 \\
        0  & x = 0 \\
        -1 & x < 0
    \end{cases}
\end{equation*}
\subsection{Unit Step Function}
The unit step (or Heaviside) function is defined as:
\begin{equation*}
    u_s\left( t \right) = \begin{cases}
        1 & t \geq 0 \\
        0 & t < 0
    \end{cases}
\end{equation*}
\subsection{Triangular Function}
The triangular function is defined as:
\begin{equation*}
    \Lambda_T\left( t \right) = \begin{cases}
        1 - \frac{\abs*{t}}{T} & \abs*{t} < T     \\
        0                      & \text{otherwise}
    \end{cases}
\end{equation*}
\subsection{Gate Function}
The gate (or rectangle) function is defined as:
\begin{equation*}
    \mathrm{\Pi}_T\left( t \right) = \mathrm{rect}\left( \frac{t}{T} \right) = \begin{cases}
        1 & -\frac{T}{2} \leq t \leq \frac{T}{2} \\
        0 & \text{otherwise}
    \end{cases}
\end{equation*}
\end{document}
